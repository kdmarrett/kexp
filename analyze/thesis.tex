\documentclass[12pt]{article}
\usepackage{setspace}
\usepackage{fullpage}
%\usepackage[T1]{fontenc}
%\usepackage[utf8]{inputenc}
\usepackage{authblk}
\usepackage{graphicx}
\usepackage{amssymb}
\usepackage[font=footnotesize]{caption}
\usepackage[superscript]{cite}
\usepackage[english]{babel}
%\graphicspath{ {./figures/} }
\usepackage{float}
% \floatstyle{boxed}
\restylefloat{figure}
\doublespacing
\title{\LARGE A User Centered Approach for Auditory P300 Brain Computer Interface Design}
\author[1]{\footnotesize Karl Marrett}
\author[2]{\footnotesize Mark Wronkiewicz}
\author[3]{\footnotesize Adrian Lee}
%\author[4]{\footnotesize Michael  Tangermann}
\affil[1]{\scriptsize Department of Neurobiology, University of Washington}
\affil[2]{\scriptsize Department of Neurobiology and Behavior, University of Washington}
\affil[3]{\scriptsize Department of Speech and Hearing Science, University of Washington}
%\affil[4]{\scriptsize Department of Computer Science, University of Freiburg} 
\date{}
\begin{document}
%TODO get a bibliography going
\begin{Abstract}

In order to communicate, patients with total loss of muscle control
including eye movement can rely on brain computer interfaces (BCI).
BCIs utilize the P300 response, an evoked-event related potential of
the brain recorded using electroencephalography. Typically, P300 spellers rely on
visual input.  However, due to the auditory system's unique ability
of spatial selective attention of an auditory stream, new auditory
speller paradigms that can aid a listener's ability to selectively
attend may increase the bit rate of communication over the current
standard or be more user friendly.  Various strategies for improving
auditory P300-based speller designs rely on multiple factors, e.g.,
effectiveness of spatial selective attention. These cues are
incorporated into modern auditory BCI systems with an understanding
of the effect on bit rate but without a heuristic for understanding
the effect on usability. This study takes three sample auditory
paradigms of varying presumed difficulty and examines their relative
cognitive load via pupillometry and the subjective survey of the
NASA Task Load Index. One condition has selections that are,
alphabetically ordered, another condition has letters fixed in an
order that separates phonemically similar letters to reduce errors,
and the final condition segregates similar letters as well but
changes the ordering throughout the trial randomly.  If subject's
perception of difficulty corroborates with their pupil responses,
then this strategy outlines a general method for assessing task
difficulty.  For thirteen Native English speakers
with no hearing difficulties, results showed that subjects rated a fixed
order presentation and random ordered presentation as more difficult
than the alphabetic condition.  In addition, compelling correlations
were found between the subject pupil size means and subjective
cognitive load for the alphabetic and fixed-order conditions
although not significant (p=.07 for both).  Relative pupil sizes
showed that the fixed-order mean was greater than the random
condition (p=.06).No significant differences were found for the
subject's raw ability to attend to the letters for the three
conditions. By combining objective and subjective measures of
usability, this project offers a strategy for comparing cognitive
load for not only the end user community of individuals who rely on
speller systems, but in other areas for the BCI community as a
whole.

\end{Abstract}

\maketitle

\Section{Introduction and Background}
   
\paragraph{Spatial selective attention} 
Humans have the ability in a crowded auditory environment to
selectively listen to different sounds and segregate different
sources into different \lq streams\rq (i.e. the \lq cocktail
party effect\rq) in what is known as auditory scene analysis
\cite{Bregman1994}. An auditory stream can be defined as a group of
successive or simultaneous sound elements that yield the percept of
a single object or source\cite{Moore2012}. These streams can be formed by
discriminating features such as spatial location or
temporal coherence and in general the similarity and proximity of
sounds in sequence determines the perceived auditory streams in an
auditory environment\cite{Bregman1994}.  Listener's can consciously
shift attention among these elements, and selective attention in
this manner has been shown to improve over time\cite{Best2008}.

\paragraph{P300 Response} 
The P300 response is an event related
potential (ERP) characterized by a positive voltage deflection
around 300 ms after a rare or unexpected stimulus typically recorded
using electroencephalography (EEG). In P300 tasks, a subject is
typically presented with a series of stimuli that fall into one of
two classes target or non-target.  The target class must ahve lower
probability of occurring than the other (typically one in four
probability.  By slectively attending the sitmuli, the subject
classifies the stimuli as target ornon-target \cite{Wolpaw2012}.
ERPs such as the P300 response can be used for gauging user intenet
in an auditory brain computer interface (BCI) device.

P300 amplitude increases with increased absolute
time gaps between oddball presentation \cite{GONSALVEZ2002}, and
amplitude is also affected by moment to moment changes in the
probability a stimulus will appear, where higher probabilities
result in diminished amplitude of responses \cite{Donchin1981}.
 Also, when users performed two tasks simultaneously that elicited
oddball responses, when one of the tasks increased in difficulty and
required more user attention, the P300 amplitude associated with the
more difficult task had larger more salient oddball responses.
\cite{Sirevaag1989}.  Furthermore, an auditory paradigm with an
oddball probability of 25\% is likely to yield a P300 response
\cite{Nijboer2008}. 

\paragraph{Auditory P300 Spellers} 
Since the P300 response is attention mediated, the neural responses
can be leveraged in creating an speller BCI device. These speller
devices are typically used for patients with diseases such as
amyotrophic lateral sclerosis where motor ability is so severely
impaired that brain signals remain the last means of
communication\cite{Nijboer2008}.  Visual spellers are most commonly
used but due to the diversity in the population of users (i.e.
blindness, eye-movement loss, eyelid-movement loss) recent research
has focused on replicating the performance of visual P300 spellers
in the auditory domain.  In addition, although the goal of improving
the raw rate at which information is transferred (bit rate) of
the P300 speller is a several decade old problem, basic
psychophysics research is still vital to carry out in parallel with
new advancement in classification methods.

Recent research has shown that even known ordering of stimuli can
produce a classifiable responses.  In a new auditory speller known
as the “charStreamer" paradigm, letters are ordered
alphabetically \cite{Hohne2014}.  While alphabetic ordering does not
provide as robust a signal as the classical oddball paradigm in this
filtering method, it still retains some statistical structure that
new classification methods could extract \cite{Hohne2014}.  

%TODO are there other studies that segregate letters based on simi?

%\cite{Schreuder2010a,Schreuder2011,Hohne2011a}. 
%Although the P300 response is deemed the oddball response because stimuli are
%unexpected, the sequential and predictable letters in this paradigm
%do in fact produce a classifiable brain response 

\paragraph{Question}

Previous experiments show that, to some degree increasing the number
of selections presented in a trial decreases performance due to
informational masking \cite{Maddox2012a}. Therefore, in order to
create a high bit rate auditory speller paradigm, higher selections
must be added while minimizing the factors of increased trial time
and informational masking. 

This experiment tests the effects of alphabetic ordering, the
effects of phonemic similarities in letters, and the
effects of dynamic vs. ordered selections.  In particular we explore 
which of these features determine listener's ability to discriminate
target selections, how these features correlate with cognitive load,
and the ability to corroborate these percepts of cognitive load with
pupillometry data.

The conditions test the usability of some of the new features
incorporated into modern auditory BCI systems which are generally
added with the assumption but without the scientific confirmation
that they aid usability.  Cognitive load is assessed objectively via
pupillometry which is a corollary of the attention and brain
activation in a task as well as via subjective assessment using the
NASA Task Load Index survey that measures perceived workload
\cite{Zickler2013a}.  
%TODO is this the eye window to the brain

%Essentially, the key tradeoff this study
%explores is selections per time versus accuracy in discriminating
%target selections. 

%The goal of this study is to evaluate the accuracy and cognitive load of several
%example auditory BCI features.  

%The major aspects can be separated into:
%1) developing the human computer interaction component assessing the
%psychophysics limitations of current auditory BCI systems 2)
%assessing the generalized objective and subjective assessment of
%cognitive load in users of a BCI paradigm.  The
%psychophysics aim of this study is used to identify listener’s build
%up and steady state cognitive load involved in selectively attending
%a letter in an auditory BCI environment across 3 conditions.  

\section{Methods}

Thirteen subjects took part in this experiment.  All subjects were
Native English speakers with no hearing difficulties.  Participants
had pure-tone threshold in both ears within 20 dB of normal-hearing
thresholds at octave frequencies between 250 and 8000 Hz.  All
subjects gave informed consent to participate in the study as
overseen by the University of Washington Institutional Review Board.

Stimuli consisted of all letters of the alphabet plus the additional
commands of 'Pause', 'Space', 'Read', and 'Delete' for a total of
thirty selections. The selections were spoken recordings of two
similar sounding male speakers.  These selections were divided by
virtual spatial location (left, middle, and right) into three groups
as shown in Figure ~\ref{screenshot}.  These selections were
monotonized using Praat software and processed with pseudo-anechoic
head-related transfer functions (recorded from a KEMAR manikin at 1
m).  All stimuli were generated at a sampling rate of 24414 Hz and
sent to Tucker-Davis Technologies hardware for digital-to-analog
conversion and attenuation, and then presented over in-ear
headphones (Etymotic Research ER-2).

Stimuli fell into three conditions alphabetic, fixed-order, and
random.  The three conditions were chosen to isolate a feature of
auditory spellers that may have an effect on cognitive load.  The
ordering and spatial distribution of letters varied by condition as
explained in Table \ref{conditionTable}.  In the alphabetic
condition, letters are assigned a spatial location in alphabetic
order.  The alphabetic ordering was chosen since it is generally
intuitive and engrained for Native English speakers.  This condition
also closely matches the charStreamer paradigm proposed by Hohne et
al.  2014\cite{Hohne2014}.  In the fixed-order condition, letters
are assigned a spatial location so as to separate phonemically
similar letters.  Throughout trial of this condition, the ordering
of the letters remained constant.  In the random condition, letters
are assigned a spatial location in the same way as the fixed-order
condition only throughout the trial, the ordering of the letters
changed, meaning subjects had no cue as to when the target letter
would occur. 

The three circles in the example visual primer (Figure
\ref{screenshot}), respresent the three groups of a trial.  Each
group has ten selections and a location.  The selections within a
group are spoken sequentially with a 450 ms lag between letters.
The complete recitation of all letters in the group thus lasts 4.5
seconds and is termed cycle.  A trial consisted of five repeated
cycles. Subjects were informed to only listen to the target
group/location for the trial, subjects were expected to ignore the
other two groups at the different locations which played
simultaneously. This means all thirty selections were presented to
the subject every 4.5 seconds but due to selective attention they
were only attending the recitation of ten possible selections. All
the groups were asynchronized from the two others by 150 ms such
that no letters would ever have coinciding onsets during a trial.

The time course of a trial is diagrammed in Figure \ref{trialtrace}.
Each trial consisted of a priming period lasting the first seven
seconds.  During this priming period, the spatial locations of the
letters were displayed for reference.  The target letter was
highlighted in green and also spoken at its respective location so
that the listener knew before the trial began which letter
group/location to listen to for the entirety of the trial. At seven
seconds, the visual primer disappeared and subjects began fixation on
a small white dot at the center of the screen for the remainder of
the trial.  The task lasted from thirteen seconds, when the spoken
letters began, to thirty-six seconds, the end of the trial.
Approximately half of all trials contained
two targets, the other half, one. The experiment consisted of
twenty-seven trials per condition with nine trials at each
spatial location.  

To keep attention and gauge accuracy, after each trial subjects
were prompted to report the targets that occurred in the trial. To test listener's raw ability
to discriminate letters among the conditions, subjects were given a
target letter and prompted at the end of each trial to enter the
occurrences of that trial.  

To survey subjects experience of cognitive load subjectively, the
NASA task load index (TLX) was performed after completion of the
experiment.  The survey consisted of two sections which was
completed independently for each condition.  The first section of
the survey prompted subjects to rate the condition's difficulty from
1 to 9 on 6 factors: mental demand, performance-based demand, effort
based demand, temporal demand, frustration, and physical demand. In
the second section, subjects were given pairs of factor titles and
asked to choose which was the more important contributor to
workload.  From these comparisons a weighting was computed for each
factor.  

%All distributions were assumed to be normal for significance testing
%with the Pearson correlation coefficient and the paired t-test. 

One subject that communicated that they had 
misunderstood the task directions by listening at all locations for
the letter after completing the experiment
was removed from the data set. Written responses were not elicited
during the survey section, however, subjects were made aware that
comments they made regarding the subject may be recorded and used.

More technical information and source files with documentation can
be found online at https://github.com/kdmarrett/kexp.

\begin{figure}[p]
  \centering
  \includegraphics[width=0.8\textwidth]{1screenshot}
  \caption{ An example of the visual primer shown
      before each trial.  This primer depicts the phonemically
      displaced ordering characteristic of the primers of the
      Fixed-Order and Random condition. The three circles represent
  the three letter groups of the trial.  The target letter is
  highlighted in green.  The clockwise ordering of
  the letters (first letter at the bottom) indicates the sequence the selections will be spoken
  in during any given cycle.  Note that this ordering is only
  consistent in the alphabetic and fixed-order conditions.}
  \label{screenshot}
\end{figure}

\begin{figure}[p]
  \centering
  \includegraphics[width=0.8\textwidth]{Meanbasecorrectedpupilsize_barplot}
  \caption{Shows the across subject means along with the across
  subject standard error of theh mean pupil size for the second
  through fourth cycle of the task period. The y axis indicates the
  relative pupil size.  This number is in arbitrary units (AU) and
  is proportional to the number of pixels identified as pupil during
  the trial recording.}
  \label{psBarplot}
\end{figure}

\begin{figure}[p]
  \centering
  \includegraphics[width=0.8\textwidth]{Weightedcognitiveloadsurvey_barplot}
  \caption{Depicts the across subject means and standard errors of
  the scores for each condition in the three conditions.  }
  \label{cogLoad}
\end{figure}

\begin{figure}[p]
  \centering
  \includegraphics[width=0.8\textwidth]{Basecorrected_trace}
  \caption{Depicts the mean pupil size over the course of a trial
  for each condition.  The colored fill indicates the standard error
  for the 13 subjects for the respective condition.  The y axis indicates the
  relative pupil size.  This number is in arbitrary units (AU) and
  is proportional to the number of pixels identified as pupil during
  the trial recording.}
  \label{trialtrace}
\end{figure}

\begin{figure}[p]
  \centering
  \includegraphics[width=0.8\textwidth]{Accuracy_barplot}
  \caption{Depicts the global averages across subjects for each
  condition.  Individual subject scores are shown in ball and stick.  Error bars
  indicate the standard error for the 13 subjects.}
  \label{accuracy}
\end{figure}

\begin{table}[p]
  \centering
  \begin{tabular}{ | l | c | c | p{4mm} |}
    \hline
    Condition & Alphabetic & Fixed-Order & Random \\
    \hline
    Alphabetic Ordering & \checkmark & & \\ 
    \hline
    phonemically Displaced Ordering &  & \checkmark & \checkmark  \\ 
    \hline
    Changing Letter Ordering &  & & \checkmark  \\ 
    \hline
  \end{tabular}
  \caption{A schematic illustrating the features of the three
      condition types.  In the alphabetic condition, all letters are
      assigned alphabetically and retain this ordering throughout
      the trial.  In the fixed-order condition, the ordering
      maximizes displacement of phonemically similar letters (i.e.
      "B" and "C" will not appear next to eachother) and this
      ordering is retained throughout the trial.  In the random
      condition, letters are assigned to displace similar letters
      but the letter ordering over the course of the trial.}
  \label{conditionTable}
\end{table}

\Section{Results}

\paragraph{Behavioral}
No significant difference was found for subjects ability to
discriminate the target letter among the three conditions.
The mean accuracy for the Alphabetic, Fixed-Order, and Random
condition were \% 83.77 (st. error \pm2.13), 87.89 (st. error \pm3.27),
and 83.61 (st. error \pm  .0290) respectively.  The two-tailed paired t-testing among 
the Alphabetic and fixed-order condition yielded a p-value of .1469,
among alphabetic and random, .9627, and between fixed-order and
random, .1253.  None of the conditions met the p<.05 criterion needed
to reject the null hypothesis that means across the conditions are
equal. These results are summarized in Figure \ref{accuracy}.

\paragraph{Physiological}
%TODO mention baseline correction in the methods
%TODO mention AU is arbitrary units
%TODO tie into the figures
%TODO check all these values
The across subject mean for the baseline corrected pupil responses
(\pm standard error of the subject mean) was 303.3472 (\pm 0.8109),
377.9681 (\pm 1.1683), and 235.2736 (\pm 0.6368) for the alphabetic,
fixed-order, and random conditions respectively.
The paired t-test
yielded 0.2259 for Alphabetic and Fixed-Order, 0.4533 for Alphabetic
and Random, and 0.0669 for Fixed-Order and Random conditions.

%TODO check the deets on the window time
When we instead look at the areas around targets, the distinction
between the random and fixed-order becomes more significant.  Taking
the baseline corrected mean of the window from the target letter
onset to 1.5 seconds later yields values of 338.2616 (\pm 0.8811),
413.3336 (\pm 0.8009), and 215.5003 (\pm 0.8786) for the alphabetic,
fixed-order, and random conditions respectively.  Comparing p-values
yields 0.2578 for the Alphabetic and Fixed-Order, 0.2408 for the
Alphabetic and Random, and 0.0437 for the Fixed-Order and Random conditions.

Taking the maximum value within the target window of the baseline
corrected cross subject mean yielded  values of 429.0366 (\pm
151.5417), 510.4105 (\pm 142.6575), and 322.7502 (\pm 130.2307) for
the alphabetic, fixed-order, and random conditions respectively.
The cross condition p-values were 0.2628 for Alphabetic and
Fixed-Order, 0.3241 for Alphabetic and Random, 0.0513 for
Fixed-Order and Random conditions.

However the significance of this difference disappears when we
instead just at the local peak height within this window.  These
values are the across subject base corrected max values subtracted
by the mean task pupil size for that condition.  This measure gives
a sense of how much the pupil size changes with respect to the rest
of the condition. The means were 90.7750 (\pm 13.7231), 97.0769 (\pm
12.9380), and 107.2499 (\pm 14.3094) for the alphabetic,
fixed-order, and random conditions respectively.  The cross
condition p-values were 0.6805 for Alphabetic and Fixed-Order,
0.3088 for Alphabetic and Random, and 0.2926 for Fixed-Order and
Random conditions.  None of these differences are significant.


\paragraph{Subjective}

Cognitive load weighted sig. testing:
The NASA task load index survey yielded contrary results.  
The weighted score for alphabetic, fixed-order, and random
conditions respectively were 
21.5337 (\pm 0.4456), 22.2740 (\pm 0.4675), 22.8365 (\pm 0.4247).

Paired t-test p-values between the Alphabetic and Fixed-Order was
.0338, between Alphabetic and Random .0050, and between Fixed-Order
and Random .0734.  These results are summarized in Figure
\ref{cogLoad}.  Thus the cognitive load in alphabetic condition was
rated significantly less demanding than both the fixed-order and
random trials.

Cross-measure correlation
Comparing the three measures of cognitive load for a linear
correlation via the pearson coefficient, no measures were found to
be significantly correlated. However, some correlations between the
mean baseline corrected pupil size for the task period and the survey were high.
Between these two measures the Alphabetic condition had a p-value
of 0.0747.  Between the fixed-order, 0.0743, and between random, .3102
Comparisons were made for each condition
between each combination of the three measures.  All other
comparisons of conditions between either measure with accuracy yielded
p-values above .100.

\Section{Discussion and Conclusions}

\paragraph{Behavioral}
No significant effects were found between condition types for the
measure of accuracy, or the correlation of this measure with the
others.  Accuracy in discriminating the target is the only measure
relating the different conditions to their projected bit rate in an
auditory speller paradigm, since, if a subject could not hear their desired
letter it would be unlikely that a classifier would still
register the letter. From this respect, the type of task in the
experiment did not show any significant effect on the potential bit rate of
these systems. This means that the major effects of these different
design strategies may only be differences in usability. Since the
verification of accuracy introduces a new task outside of the
paradigm being tested, and it still did not show significant effects
future studies assessing cognitive load in an auditory BCI paradigm
may want to omit these forms of verification.

\paragraph{Physiological}
Although the difference between the Fixed-Order and Alphabetic
condition was not quite significant ($p=.0634$), the value is
still compelling.  Inability to resolve these conditions may be
because of two reasons.  At N=13 the subject count may be too low to
resolve the population means of the two conditions. In addition, the
pupil dynamic range for each subject was not recorded, meaning that
individuals with a large range may have been given artificially higher
weighting in the calculation of the grand mean.

Although sustained
attention may perceived as difficult, the actual cognitive
processing involved in such a task may actually be supressed as
compared to one.  In fact, if we consider the pupil size to be
proportional to mental activity, the potential difference between
pupil size for the random and fixed order condition may be
explainable.  For example, the random condition provides no
information to the listener about the occurrence of the letter other
than its spatial location so the subject essentially must passively
listen for the entire trial.  Since all subjects were native English
speakers the ordering of the alphabet was likely at least somewhat
familiar so they might have listened passively with some activity
related to checking the ordering around the target.  In the
fixed-order condition an entirely new ordering was For the most
part.  This interpretation is corroborated by one subject comment.
One subject reported that it was easier not to consider which trial
was of which condition type and instead just listen for the letter.
Although, consciously ignoring the conditions does not necessarily
mean that task difficulty or accuracy among the conditions is equal
(this subject rated all conditions equally in the survey but had
differences in accuracy and pupil size between the three
conditions).

Since subjects were made aware at the start of the experiment that
they would be scoring the relative difficulty of the three
conditions, some may have assumed the additional implicit task of
distinguishing the fixed-order and random conditions since these
conditions can also be discriminated during the trial and not by
their visual primer.  However, this claim would refute any
difference between the random and fixed-order trace means since, if
the task of discriminating the two conditions was in fact inflating
the pupil size, it would increase both conditions. 

Another thing to notice about the trace is the ramp down during the
trial for each condition.  This is unlikely to be due to any changes in
the difficulty of the task as the trial continues. This is because, although
the probability of having a target is symmetric about the entirety of
the task period of the trial (13 - 36 s), subjects know that there will be at
most two targets.  In about \%50 of all trials of
the experiment, the trial contain two targets. This means that 
in those trials with two targets when the second target occurs, the subject will likely stop
performing the task.  Note that in the period before the task
begins, although subjects are exposed to the same fixation dot the
mean pupil size is significantly lower than the task mean.

\paragraph{Subjective}
In light of the lack of significant findings in the other separate
measures, we take the subjective survey results as the relative
difficulty of each condition.  The subjective results indicate that
ordering has the largest effect on perceived work load. Despite
fixed-order having maximally displaced phonemically similar letters,
it was still scored as significantly harder than the known
alphabetic condition.  In addition, the random condition was rated
more difficult than the alphabetic ordering, indicating that there
was incentive for familiarizing oneself with the fixed-order
ordering throughout the trial and/or experiment such that one could
use the order as an informative cue.  Since there as an incentive to
learn the fixed-order condition it is likely that that the
difference in difficulty is due to the added mental activity of
learning the new ordering throughout the experiment. It is therefore
unclear whether this difference would persist over multiple days or
trials or if the workload would decrease below alphabetic level due
to displaced letters.

Sequential ordering (the charStreamer paradigm) offers a potential to
greatly increase the usability of current speller systems.

An additional complication to note is that in the
random condition the letters were distributed into the three
spatial locations such that similar sounding letters were not only
put into different groups but also where multiple letters were put
into a wheel they were distributed so as not to be contiguous.  As
an example in Figure \ref{screenshot}, X and X must be in the left
spatial location but they are placed such that they are not spoken
sequentially to avoid confusion.  This holds in the fixed-order
condition, however in the random location since the letters within
the locations change between cycles, some similar letters will
occasionally inevitably be sequential. 

Although no written responses were elicited, one subject commented
that discriminating similar sounding letters was difficult.  Another
subject commented that looking ahead while the letter was at a
different location such as left or right was more difficult.  The
feature of requiring subjects to look at the center of the screen
was an arbitrary but necessary additional cognitive load that while
was a necessity in the experiment would not necessarily exist in the
end-user application settings of an auditory speller device.

%Why usability issues are a large concern
%small population with unique clinical concerns
Why even though absolute information about which paradigm is more
usable, or accurate for the population these methods may serve as a
heuristic for judging what type of system might be more usable for
an Individual.

\Section{Acknowledgments}

We would like to thank the Computational Neuroscience Training Program
and the Center for Sensorimotor Neural Engineering for funding towards
the Brain Links - Brain Tools summer exchange. We would also like to
thank the Levinson Emerging Scholarship and the Mary Gates Research
Scholarship for support during this project.

\clearpage    
\bibliographystyle{plain}
\bibliography{kexp.bib}  
\end{document}
