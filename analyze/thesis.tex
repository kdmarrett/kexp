\documentclass[10pt]{article}
\usepackage{setspace}
\usepackage[font=footnotesize]{subcaption}
\usepackage[font=footnotesize]{caption}
%\usepackage{fullpage}
%\usepackage[T1]{fontenc}
%\usepackage[utf8]{inputenc}
\usepackage{authblk}
\usepackage{graphicx}
\usepackage{amssymb}
\usepackage[superscript]{cite}
\usepackage[english]{babel}

\usepackage[sc]{mathpazo} % Use the Palatino font
\usepackage[T1]{fontenc} % Use 8-bit encoding that has 256 glyphs
\linespread{1.05} % Line spacing - Palatino needs more space between lines
\usepackage{microtype} % Slightly tweak font spacing for aesthetics

%\usepackage[hmarginratio=1:1,top=32mm,columnsep=20pt]{geometry} % Document margins
\usepackage[top=1in, bottom=1in, right=1in, left=1in, columnsep=20pt]{geometry} % Document margins
\usepackage{multicol} % Used for the two-column layout of the document
%\usepackage[hang, small,labelfont=bf,up,textfont=it,up]{caption} % Custom captions under/above floats in tables or figures
\usepackage{booktabs} % Horizontal rules in tables
\usepackage{float} % Required for tables and figures in the multi-column environment - they need to be placed in specific locations with the [H] (e.g. \begin{table}[H])
\usepackage{hyperref} % For hyperlinks in the PDF
\usepackage{abstract} % Allows abstract customization
\renewcommand{\abstractnamefont}{\normalfont\bfseries} % Set the "Abstract" text to bold
\renewcommand{\abstracttextfont}{\normalfont\footnotesize\itshape} % Set the abstract itself to small italic text

%\usepackage{titlesec} % Allows customization of titles
%\renewcommand\thesection{\Roman{section}} % Roman numerals for the sections
%\renewcommand\thesubsection{\Alph{subsection}} % Roman numerals for subsections
%\titleformat{\section}[block]{\large\scshape\centering}{\thesection.}{1em}{} % Change the look of the section titles
%\titleformat{\subsection}[block]{\large}{\thesubsection.}{1em}{} % Change the look of the section titles

\usepackage{fancyhdr} % Headers and footers
\pagestyle{fancy} % All pages have headers and footers
%\fancyhead{} % Blank out the default header
%\fancyfoot{} % Blank out the default footer
%\fancyhead[C]{Marrett et al. $\bullet$ Neurobiology Bachelors
%Thesis $\bullet$ June 2015} % Custom header text
%\fancyfoot[RO,LE]{\thepage} % Custom footer text
%\fancyfoot{\thepage} % Custom footer text

%\graphicspath{ {./figures/} }
\usepackage{float}
% \floatstyle{boxed}
\restylefloat{figure}
\doublespacing
\title{\LARGE A User Centered Approach for Auditory P300 Brain Computer Interface Design}
\author[1]{\small Karl Marrett}
\author[2]{\small Mark Wronkiewicz}
\author[3]{\small Adrian Lee}
%\author[4]{\footnotesize Michael  Tangermann}
\affil[1]{\scriptsize Department of Neurobiology, University of Washington}
\affil[2]{\scriptsize Department of Neurobiology and Behavior, University of Washington}
\affil[3]{\scriptsize Department of Speech and Hearing Science, University of Washington}
%\affil[4]{\scriptsize Department of Computer Science, University of Freiburg} 
\date{}
\begin{document}
\maketitle
%TODO get a bibliography going kexp.bib should include Moore2012
%TODO make sig figs agree wit
%TODO results explain subj mean \pm sem at the beginning
%TODO add in paragraph for question
%TODO axis lines bolder

    %In order to communicate, patients with total loss of muscle control
    %including eye movement can rely on brain computer interfaces (BCI).
    %BCIs utilize the P300 response, an evoked-event related
    %potential of the brain recorded using
    %electroencephalography. 

  %If subject's
    %perception of difficulty corroborates with their pupil responses,
    %then this strategy outlines a general method for assessing task
    %difficulty.One condition has selections that are,
    %alphabetically ordered, another condition has letters fixed in an
    %order that separates phonemically similar letters to reduce errors,
    %and the final condition segregates similar letters as well but
    %changes the ordering throughout the trial randomly.  
    %but in other areas for the BCI community as a
        %whole.

    %Various strategies for improving
    %auditory P300-based speller designs rely on multiple factors, e.g.,
    %effectiveness of spatial selective attention. These cues are
    %incorporated into modern auditory BCI systems with some understanding
    %of the effect on bit rate but without a heuristic for understanding
    %the effect on usability. 
    
\begin{abstract}

    Attention mediated neural signals such as the P300
    response allow some patients with total loss of muscle
    control to communice via a speller device.  Typically,
    P300 spellers rely on visual input.  However, due to the
    auditory system's unique capability of spatial selective
    attention of an auditory stream, new auditory speller
    paradigms that can aid a listener's ability to selectively
    listen may increase the bit rate of communication over the
    current standard and be more user-friendly.  In three
    conditions, alphabetic ordering, fixed ordering
    (non-alphabetic), and altered ordering, we examined the
    relative cognitive load via pupillometry and a subjective
    survey known as the NASA Task Load Index.  Native English
    speaker subjects (N=13) rated the alphabetic condition as
    less difficult than the fixed order presentation and
    random condition (p=.03, p=.005 respectively).
    Correlations were found between relative pupil size and
    subjective rating of cognitive load for the alphabetic and
    fixed-order conditions although not significant (p<.09,
    p<.08 respectively).  Relative pupil sizes showed that the
    fixed-order mean was greater than the random condition
    (p<.03), especially in response to targets (p=.04). No
    significant differences or correltations were found for
    the subject's raw ability to attend to the letters among
    the three conditions. By combining objective and
    subjective measures of usability, this study offers a
    strategy for determining cognitive load for the end-user
    community of individuals who must rely on speller systems.

\end{abstract}


%\begin{multicols}{2} % Two-column layout throughout the main article text

\section{Introduction and Background}
\subsection{Spatial selective attention} 
Humans have the ability in a crowded auditory environment to
selectively listen to different sounds and segregate different
sources into different \lq streams\rq (i.e. the \lq cocktail
party effect\rq) in what is known as auditory scene analysis
\cite{Bregman1994}. An auditory stream can be defined as a group of
successive or simultaneous sound elements that yield the percept of
a single object or source\cite{Moore2012}. These streams can be formed by
discriminating features such as spatial location or
temporal coherence and in general, the similarity and proximity of
sounds in sequence determines the perceived auditory streams in an
auditory environment\cite{Bregman1994}.  Listener's can consciously
shift attention among these elements, and this manner of selective attention 
has been shown to improve over time\cite{Best2008}.

\subsection{P300 Response} 
The P300 response is an event related
potential (ERP) characterized by a positive voltage deflection
around 300 ms after a rare or unexpected stimulus, and is typically recorded
using electroencephalography (EEG). In P300 tasks, a subject is
typically presented with a series of stimuli that fall into one of
two classes: target or non-target.  The target class must have lower
probability of occurring than the other (typically a one in four
probability).  By selectively attending the stimuli, the subject
classifies the stimuli as target or non-target \cite{Wolpaw2012}.

P300 amplitude increases with increased absolute time gaps
between oddball presentation \cite{GONSALVEZ2002}. 
Amplitude is also affected by moment to moment changes in the
probability that a stimulus will appear, where higher probabilities
result in diminished amplitude of responses
\cite{Donchin1981}.  Also, when users performed two tasks
simultaneously that elicited oddball responses, when one of
the tasks increased in difficulty and required more user
attention, the P300 amplitude associated with the more
difficult task had larger, more salient oddball responses
\cite{Sirevaag1989}.  Furthermore, an auditory paradigm with
an oddball probability of 25\% is likely to yield a P300
response \cite{Nijboer2008}. 

\subsection{Auditory P300 Spellers} 
%TODO the flow of this section is off
ERPs such as the P300 response can be used for gauging user intent
in an auditory brain computer interface (BCI) device.
These speller
devices are typically used for patients with diseases such as
amyotrophic lateral sclerosis where motor ability is so severely
impaired that brain signals remain the last means of
communication\cite{Nijboer2008}.  Visual spellers are most commonly
used but due to the diversity in the population of users (i.e.
blindness, eye-movement loss, eyelid-movement loss) recent research
has focused on replicating the performance of visual P300 spellers
in the auditory domain.  In addition, although the goal of improving
the raw rate at which information is transferred (bit rate) of
the P300 speller is a problem several decade old, basic
psychophysics research is still vital to carry out in parallel with
new advancements in classification methods.

Recent research has shown that even known ordering of stimuli can
produce classifiable responses.  In a new auditory speller known
as the “charStreamer" paradigm, letters are ordered
alphabetically \cite{Hohne2014}.  While alphabetic ordering does not
provide as robust a signal as the classical oddball paradigm in this
filtering method, it still retains some structure that
new classification methods could extract \cite{Hohne2014}.  

Previous experiments show that, to some degree increasing the number
of selections presented in a trial decreases performance due to
informational masking \cite{Maddox2012a}. Therefore, in order to
create a high bit rate auditory speller paradigm, higher selections
must be added while minimizing the factors of increased trial time
and informational masking. 

%TODO are there other studies that segregate letters based on simi?
%\cite{Schreuder2010a,Schreuder2011,Hohne2011a}. 
%Although the P300 response is deemed the oddball response because stimuli are
%unexpected, the sequential and predictable letters in this paradigm
%do in fact produce a classifiable brain response 

%\subsection{Question}

This experiment tests the effects of alphabetic ordering, the
effects of phonemic similarities in letters, and the effects
of dynamic vs. ordered selections.  In particular we explore
which of these features determine a listener's ability to
discriminate target selections, how these features correlate
with cognitive load, and the ability to corroborate these
perceived cognitive load with pupillometry data.  Cognitive load is
assessed objectively via pupillometry which is a corollary of
the attention and brain activation in a task as well as a
subjective assessment using the NASA Task Load Index survey
that measures perceived workload \cite{Zickler2013a}.  The
conditions test the usability of some of the new features
incorporated into modern auditory BCI systems which are
generally added with the assumption, but without the scientific
confirmation, that they aid usability.    

The objective of this study is twofold.  One aim is to assess
different auditory paradigms using three
available measures: the accuracy in hearing a target letter which objectively tests
a subjects ability to do the task,  pupil sizes which objectively
tests cognitive load, and the NASA Task Load Index, a survey which
test cognitive load subjectively.  The second aim is to examine the
effectiveness of these three measures in discriminating cognitive
load or accuracy
between possible paradigms for future studies.

%TODO is this the eye window to the brain

%The major aspects can be separated into:
%1) developing the human computer interaction component assessing the
%psychophysics limitations of current auditory BCI systems 2)
%assessing the generalized objective and subjective assessment of
%cognitive load in users of a BCI paradigm.  The
%psychophysics aim of this study is used to identify listener’s build
%up and steady state cognitive load involved in selectively attending
%a letter in an auditory BCI environment across 3 conditions.  

\begin{table}[ht]
  \centering
  \begin{tabular}{ | l | c | c | c |}
    \hline
    Condition & Alphabetic & Fixed-Order & Random \\
    \hline
    Alphabetic ordering & \checkmark & & \\ 
    \hline
    Phonemically displaced grouping &  & \checkmark & \checkmark  \\ 
    \hline
    Changing letter ordering &  & & \checkmark  \\ 
    \hline
  \end{tabular}
  \caption{A schematic illustrating the features of the three
      condition types.  In the alphabetic condition, all letters are
      assigned alphabetically and retain this ordering throughout
      the trial.  In the fixed-order condition, the ordering
      maximizes displacement of phonemically similar letters (i.e.
      B and E will not appear immediately after one another) and this
      ordering is retained throughout the trial.  In the random
      condition, letters are also assigned to displace similar letters
      but the letter sequence within a group changes sequence
      after each cycle of the trial.}
  \label{conditionTable}
\end{table}

\section{Methods}

Thirteen subjects took part in this experiment.  All subjects were
Native English speakers with no hearing difficulties.  Participants
had pure-tone threshold in both ears within 20 dB of normal-hearing
thresholds at octave frequencies between 250 and 8000 Hz.  All
subjects gave informed consent to participate in the study as
overseen by the University of Washington Institutional Review Board.

Stimuli consisted of all letters of the alphabet plus the
additional commands of 'Pause', 'Space', 'Read', and 'Delete'
for a total of thirty selections.  These four commands were
always placed in the right spatial group regardless of trial
condition.  The selections were spoken recordings of two
similar sounding male speakers.  These selections were divided
by virtual spatial location into three groups (left, middle,
and right) as shown in Figure \ref{screenshot}.  These
selections were monotonized using Praat software and processed
with pseudo-anechoic head-related transfer functions (recorded
from a KEMAR manikin at 1 m).  All selections were trimmed to
410 ms in length.  All stimuli were generated at a
sampling rate of 24414 Hz and sent to Tucker-Davis
Technologies hardware for digital-to-analog conversion and
attenuation, and then presented over in-ear headphones
(Etymotic Research ER-2).

Stimuli fell into three conditions alphabetic, fixed-order,
and random.  The three conditions were chosen to isolate a
feature of auditory spellers that may have an effect on
cognitive load.  The ordering and spatial distribution of
letters varied by condition as explained in Table
\ref{conditionTable}.  In the alphabetic condition, letters
were assigned a spatial location in alphabetic order.  The
alphabetic ordering was chosen since it is generally intuitive
and engrained in native English speakers.  This condition
also closely matches the charStreamer paradigm proposed by
Hohne et al.  2014\cite{Hohne2014}.  In the fixed-order
condition, letters were assigned to a spatial location so as to
separate phonemically similar letters.  Throughout the trial
of this condition, the ordering of the letters remained
constant.  In the random condition, similar letters were
distributed in the same way as the fixed-order condition only
throughout the trial, the ordering of the letters changed,
meaning subjects had no cue as to when the target letter would
occur. 

\begin{figure}[t]
  \centering
  \includegraphics[width=0.4\textwidth]{2screenshot}
  \caption{ An example of the visual primer shown before each
      trial.  This primer depicts the phonemically displaced
      letter ordering characteristic of the primers of the
      fixed-order condition and the random condition. The three circles
      represent the three letter groups of the trial.  The
      target letter is highlighted in green.  The clockwise
      ordering of the letters (first letter at the bottom)
      indicates the sequence the selections will be spoken in
  during any given cycle.  Note that this ordering is only
  retained past the first cycle of letters in the alphabetic
  and fixed-order conditions.}
  \label{screenshot}
\end{figure}

The three circles in the example visual primer (Figure
\ref{screenshot}), represent the three groups of a trial.
Each group has ten selections and a location.  The selections
within a group were spoken sequentially with a 450 ms lag
between letters.  The complete recitation of all letters in
the group lasted 4.5 s and was termed a cycle.  A trial
consisted of five repeated cycles. Subjects were informed to
only listen to the target spatial location group for the
trial, and were expected to ignore the other two groups at the
different locations which played simultaneously. Since all
three groups were simultaneous, all thirty selections were
presented to the subject every 4.5 s, but by selectively
attending the target group, they only attended the recitation
of ten selections. All the groups were asynchronized from the
two others by 150 ms such that no letters in separate groups
would ever have coinciding onsets during a trial.
%to bin the targets into separate cycles 

The time course of a trial is diagrammed in Figure \ref{trialtrace}.
Each trial consisted of a priming period of the first seven seconds.
During this priming period, the spatial locations of the letters
were displayed for reference.  The target letter was highlighted in
green and also spoken at its respective location so that the
listener knew before the trial began which spatial group location to
listen to for the entirety of the trial. At seven s, the visual
primer disappeared and subjects began fixation on a small (~5 mm
diameter) white dot at the center of the screen for the remainder of
the trial.  The task lasted from 13 s, when the spoken letters
began, to 36 s, the end of the trial.  Between each trial there was a rest
period with no visual, or auditory stimuli besides the background
white noise.  In this period, the screen retained the same luminance
as during the trial but without a fixation dot. This baseline period
lasted 2 s.  All pupil data was baseline-corrected by
subtracting all samples of the trial by the average value during
this baseline period.  All pupil data is thus in relative arbitrary units
(AU) since the data processing strategy emphasized relative
changesin pupil size.  The raw pupillometry values were proportional
to the number of pixels of the pupil in the infrared recordings.
Approximately half of all trials contained two targets, the other
half, one. To test the listener's raw ability to discriminate letters
among the conditions, subjects were prompted at the end of each
trial to enter the target occurrences, one or two. Each cycle either
contained the target letter or the target was substituted with
another letter from within the same spatial group location.

The experiment consisted of 27 trials per condition
with 9 trials at each spatial location.  The experiment
lasted last less than 2 h with breaks.   For each
subject all 27 trials per condition were averaged for accuracy
and pupil size.  From this, the across-subject mean and
standard errors were computed.  In calculating pupillometry
means, only the second (17.5 s) through fourth cycle (31 s) of
the task window were counted since the fixed-order and random
conditions were only distinguishable after the first cycle
when the letter ordering changed.

To assess the subject's experience of cognitive load subjectively,
the NASA task load index (TLX) was performed after completion
of the experiment\cite{kubler}.  The survey consisted of two
sections which subjects completed separately for each
condition.  The first section of the survey prompted subjects
to rate the condition's difficulty from 1 to 9 on 6 factors:
mental demand, performance-based demand, effort-based demand,
temporal demand, frustration, and physical demand. In the
second section, subjects were given pairs of these factor
titles and asked to choose which was the more important
contributor to their experience of workload.  From these
comparisons, a weighting was computed for each factor.
Written responses were not elicited during the survey section,
however, subjects were made aware that comments they made
regarding the subject may be recorded and used.

%One subject communicated that they had 
%misunderstood the task directions by listening at all locations for
%the letter after completing the experiment.  This subject
%was removed from the data set. 

%All distributions were assumed to be normal for significance testing
%with the Pearson correlation coefficient and the paired t-test. 

%More technical information and source files with documentation can
%be found online at https://github.com/kdmarrett/kexp.

\begin{figure}[ht]
  \centering
  \includegraphics[width=0.6\textwidth]{Basecorrected_trace}
  \caption{Depicts the across-subject mean pupil size over the course of a trial
  for each condition.  The colored fill indicates the standard error
  for the respective condition.  The y-axis indicates the
  relative pupil size in arbitrary units (AU).}
  \label{trialtrace}
\end{figure}

\section{Results}

We set the criterion to be p<.05 in order to reject the null
hypothesis that the across-subject means were equal. P-values
below this threshold were deemed significant.

\subsection{Behavioral}

\begin{figure}[t]
    \centering
    \begin{subfigure}[t]{.25\linewidth}
        \centering
      \includegraphics[width=1.2\textwidth]{Accuracy_barplot}
      \caption{Depicts the objective measure of raw ability of subjects to attend
          to a letter in the various conditions.  No
          differences were found to be significant. }
      \label{accuracy}
    \end{subfigure}
    \qquad
    \begin{subfigure}[t]{.25\linewidth}
      \centering
      \includegraphics[width=1.2\textwidth]{Meanbasecorrectedpupilsize_barplot}
      \caption{Shows the objective measure of the mean pupil size for the second
      through fourth cycle of the task period. The y-axis indicates the
  relative pupil size in arbitrary units (AU).} 
      \label{psBarplot}
    \end{subfigure}
    \qquad
    \begin{subfigure}[t]{.25\linewidth}
      \centering
      \includegraphics[width=1.2\textwidth]{Weightedcognitiveloadsurvey_barplot}
      \caption{The results for the subjective measure of cognitive
  load.  The fixed-order and random condition were rated
  as significantly more difficult than the alphabetic
  condition. The observed differences between the fixed-order
  and random condition were not significant.}
      \label{cogLoad}
    \end{subfigure}
  \caption{Results for across-subject means for each condition
      between the subject and objective measures. Individual
  subject means are shown in ball stick and error bars
  indicate the standard error of the subject means.}
\end{figure}

No significant difference was found for subjects ability to
discriminate the target letter among the three conditions.
The across-subject mean accuracy ($\pm$ the standard error of
the subject means) for the alphabetic, fixed-order, and random
condition were $83.8 \% (\pm2.1)$, $87.9 \% (\pm3.3)$,
and $83.6 (\pm 2.9)$ respectively.  The two-tailed paired t-testing among 
the alphabetic and fixed-order condition yielded a p-value of .15,
among alphabetic and random .96, and between fixed-order and
random .13.  None of the conditions met the p<.05 criterion needed
to reject the null hypothesis, that means across the conditions had
equal accuracies. These results are summarized in Figure \ref{accuracy}.

\subsection{Physiological}
%TODO tie into the figures
%TODO check all these values

The results for the mean pupil size are summarized in Figure \ref{accuracy}.
The across-subject mean for the baseline-corrected pupil responses
($\pm$ standard error of the subject mean) was $ 302.73 (\pm
0.48)$,
$400.52 (\pm .60)$, and $179.91 (\pm 0.73)$ for the alphabetic,
fixed-order, and random conditions respectively.
The significance of the paired t-test
yielded 0.21 for alphabetic and fixed-order, 0.23 for alphabetic
and random, and 0.026 for fixed-order and random conditions.

%TODO explain all criteria of significance at the beginning of
%the results section
When we instead look at the time windows around targets, the distinction
between the random and fixed-order is also strong.  Taking
the baseline-corrected mean of the window from the target letter
onset to 2.5 s later yields values of $338.26 (\pm 0.88)$,
$413.33 (\pm 0.80)$, and $215.50 (\pm 0.88)$ for the alphabetic,
fixed-order, and random conditions respectively.  Comparing p-values
yields 0.2578 for the alphabetic and fixed-order, 0.2408 for the
alphabetic and random, and 0.0437 for the fixed-order and random conditions.
We consider the difference in peaks between at fixed-order and
random to be significant.

Taking the maximum value within the target window of the baseline
corrected cross subject mean yielded  values of $429.04 (\pm
151.54)$, $510.41 (\pm 142.66)$, and $322.75 (\pm 130.23)$ for
the alphabetic, fixed-order, and random conditions respectively.
The cross condition p-values were 0.2628 for alphabetic and
fixed-order, 0.3241 for alphabetic and random, and 0.0513 for
fixed-order and random conditions which we also consider to be significant.

However the significance of this difference disappears when we
instead just at the local peak height within this window.  These
values are the across-subject base corrected max values subtracted
by the mean task pupil size for that condition.  This measure gives
a sense of how much the pupil size changes with respect to the rest
of the condition. The means were $90.78 (\pm 13.72)$, $97.08 (\pm
12.94)$, and $107.25 (\pm 14.30)$ for the alphabetic,
fixed-order, and random conditions respectively.  The cross
condition p-values were 0.68 for alphabetic and fixed-order,
0.30 for alphabetic and random, and 0.29 for fixed-order and
random conditions.  None of these differences are significant.


\subsection{Subjective}

The NASA task load index survey yielded contrasting results than
those suggested by the pupillometry data.  
The weighted score for alphabetic, fixed-order, and random
conditions were 
$21.53 (\pm 0.44)$, $22.27 (\pm 0.46)$, $22.83 (\pm
0.42)$ respectively.

Paired t-test p-values between the alphabetic and fixed-order was
.038, between alphabetic and random .0050, and between fixed-order
and random .073.  These results are summarized in Figure
\ref{cogLoad}.  Thus the cognitive load in alphabetic condition was
rated significantly less demanding than both the fixed-order and
random trials.

\subsection{Cross-measure correlation} 
When the three measures
(accuracy in hearing the target letter, pupil size, and the NASA TLX
cognitive load survey) were tested for significance of pearson's
correlation, no measures were significantly correlated.  However,
the correlations between pupil size for the task period and the
survey were high.  Between pupillometry and the survey,
the alphabetic condition had a p-value of 0.09.  Between the
fixed-order 0.08, and between random .3102.  All other comparisons
for each condition among the the three measures of accuracy, pupillometry, and survey
yielded p-values above .100.

\section{Discussion and Conclusions}

\subsection{Behavioral}

No significant effects were found between condition types for
the measure of accuracy, or the correlation of this measure
with the others.  Accuracy in discriminating the target was the
only measure relating the different conditions to their
projected bit rate in an auditory speller paradigm, since, if
a subject can not hear their desired letter, it would be
unlikely that a classifier would still register the letter.
From this respect, the type of task in the experiment did not
show a significant effect on the potential bit rate of these
systems. This means that the major effects of these different
auditory design strategies may only be differences in usability or
perceived task load. Since
the verification of accuracy introduces a new task outside of
the paradigm being tested, and it still did not show
significant effects, it would be more effective for future
studies assessing cognitive load in auditory BCI paradigms
to test the accuracy of classificiation directly in a real
speller setting via classification of the EEG recorded P300
signals.

\subsection{Physiological}
%At N=13 the subject count may be too low to
%resolve the population means of the two conditions. 
%TODO what's the correlation with target max, target mean?
Although the difference between the fixed-order and alphabetic
condition was significant ($p<.03$), the spread did not
distinguish the alphabetic condition from the others. Inability to resolve these conditions may be
because the
pupil dynamic range for each subject was not recorded, meaning that
individuals with a large range may have been given artificially higher
weighting in the calculation of the grand mean.

Although sustained attention may perceived as difficult, the
actual cognitive processing involved in such a task may
be suppressed as compared to one where, for example, a
subject is incentivized to memorize a new letter ordering.  In
fact, if we consider the pupil size to be proportional to
mental activity, the potential difference between pupil size
for the random and fixed order condition may be explainable.
For example, the random condition provides no information to
the listener about the occurrence of the letter other than its
spatial location so the subject essentially must passively
listen for the entire trial.  Since all subjects were native
English speakers the ordering of the alphabet was 
familiar, suggesting that they may have been passive listeners
with some activity related to checking the ordering around the
target in the alphabetic condition.  However, in the fixed-order condition, an entirely new ordering
was assigned.  

Since subjects were made aware at the start of the experiment that
they would be scoring the relative difficulty of the three
conditions, some may have assumed the additional implicit task of
distinguishing the fixed-order and random conditions since these
conditions can also be discriminated during the trial and not by
their visual primer.  
This interpretation is corroborated by
one subject who reported that it was easier
not to consider which trial was of which condition type and
instead just listen for the letter.  Although, consciously
ignoring the conditions does not necessarily mean that task
difficulty or accuracy among the conditions is equal (this
subject rated all conditions equally in the survey but had
differences in accuracy and pupil size between the three
conditions).
However, this claim would refute any
difference between the random and fixed-order trace means since, if
the task of discriminating the two conditions was in fact inflating
the pupil size, it would have likely increased pupil
means in both conditions. 

Another thing to notice about the trace is the ramp down during the
trial for each condition.  This is unlikely to be due to any changes in
the difficulty of the task as the trial continues. This is because, although
the probability of having a target is symmetric about the entirety of
the task period of the trial (13 - 36 s), subjects know that there will be at
most two targets.  In about \%50 of all trials of
the experiment, the trial contain two targets. This means that 
in those trials with two targets, that when the second target occurs, the subject will likely stop
performing the task.  Note that in the period before the task
begins, although subjects are exposed to the same fixation dot, the
mean pupil size is significantly lower than the task mean.  
Having the possibility of up to 5 targets (one during each
cycle with no replacements) may be a more effective strategy
for assessing steady state load because it ensures that all
subjects are performing the task for the entirety of the trial.

\subsection{Subjective}
In light of the lack of significant findings in the other separate
measures and the high correlations with the pupillometry data, we take the subjective survey results as the ground
truth for the relative
difficulty of each condition.  The subjective results indicate that
ordering has the largest effect on perceived work load. Despite
fixed-order having maximally displaced phonemically similar letters,
it was still scored as significantly harder than the known
alphabetic condition.  In addition, the random condition was rated
more difficult than the alphabetic ordering, indicating that there
was incentive for familiarizing oneself with the fixed-order
ordering throughout the trial and/or experiment such that one could
use the order as an informative cue.  Since there as an incentive to
learn the fixed-order condition, it is likely that that the
difference in difficulty is due to the added mental activity of
learning the new ordering throughout the experiment. It is therefore
unclear whether this difference would persist over multiple days or
trials or if the workload would decrease below alphabetic level due
to displaced letters.

Sequential ordering (the charStreamer paradigm) offers a potential to
greatly increase the usability of current speller systems.

One complication to note is that in the random condition the
letters were distributed into the three spatial group
locations such that similar sounding letters were, as much as
possible, put into different groups. When multiple letters had
to be put into the same spatial group location they were
sequenced so as not to be spoken contiguously.  As an example
in Figure \ref{screenshot}, B and E must be in the left
spatial location but they are placed such that they are not
spoken sequentially to avoid confusion.  This holds in the
fixed-order condition, however in the random location, some
similar letters will occasionally inevitably be spoken contiguously since
the letters within the locations change between cycles.
Although the subject could not use timing cues in the
random condition, adjacent similar letters may not have made a substantial
difference in the outcome. 

%subject commented
%that discriminating similar sounding letters was difficult.  Another

Although no written responses were elicited, one subject commented that looking ahead while the letter was at a
different location such as left or right was more difficult.  The
feature of requiring subjects to look at the center of the screen
was an arbitrary additional cognitive task that while
was a necessity in the experiment would not necessarily exist in the
end-user application settings of an auditory speller device.

\subsection{Cross-measure correlation}
%\section{Conclusion}

These findings suggest that having a known ordering plays the
largest role in the task difficulty of an auditory speller paradigm.
While accuracy in discriminating selections provides a strong ground
truth for projected bit rate, it does not provide across-subject
differences nor is it correlated with the other measures of
cognitive load used in the study.  The pupillometry data suggests
that the population level mental activity may be higher than the
perceived workload for a task. The high levels of correlation
between pupillometry and subjective survey speaks to the strength of
both measures in assessing the cognitive load in a task.  
For the small population of individuals with unique clinical concerns who rely on
these systems, it may be more appropriate to focus on methods that
test usability on an individual level as opposed to the on a
population level.  These methods may serve as a heuristic for judging what type of system
might be more usable for an individual.

\section{Acknowledgments}
We would like to thank the Computational Neuroscience Training Program
and the Center for Sensorimotor Neural Engineering for funding towards
the Brain Links - Brain Tools summer exchange. We would also like to
thank the Levinson Emerging Scholarship and the Mary Gates Research
Scholarship for support during this project.

% \clearpage    
\begin{spacing}{.865}
%\footnotesize{
%\bibliographystyle{plain}
%\bibliography{Summer Auditory P300 Project.bib}}
\end{spacing}
%\end{multicols}
\end{document}
